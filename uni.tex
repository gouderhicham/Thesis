\documentclass[12pt]{report}
\usepackage[utf8]{inputenc}
\usepackage{graphicx}
\usepackage{titlesec}
\usepackage{xcolor}
\usepackage{fancyhdr}
\usepackage{tikz}
\usepackage{charter}
\usepackage{atbegshi} 
\usepackage[a4paper, left=1in, right=1in, bottom=0.8in, top=0.8in]{geometry}

% Border settings
\newlength{\mymargin}
\setlength{\mymargin}{0.8cm}
\renewcommand{\baselinestretch}{1.4} % 1.5x line height
\pagenumbering{gobble}  % Disables page numbering

\AtBeginShipout{%
  \AtBeginShipoutUpperLeft{%
    \begin{tikzpicture}[remember picture, overlay]
      \draw[line width=0.4pt] 
        ([xshift=\mymargin, yshift=\mymargin] current page.south west) % Bottom-left corner with margin
        rectangle 
        ([xshift=-\mymargin, yshift=-\mymargin] current page.north east); % Top-right corner with margin
    \end{tikzpicture}%
  }%
}

\begin{document}

\begin{center}
	\begin{center}\begin{minipage}
			{0.1\textwidth}

			\centering
			\hspace*{-0.7cm}
			\includegraphics[width=2.2cm]{images/uni_logo.jpg}
		\end{minipage}%
		\hfill
		\begin{minipage}{0.7\textwidth}


			\centering
			\fontsize{12}{19}\selectfont % 14pt font with 18pt line spacing
			\textbf{People's Democratic Republic of Algeria} \\
			Ministry of Higher Education and Scientific Research \\
			\textbf{Martyr Hamma Lakhdar University of El Oued} \\

		\end{minipage}%
		\hfill
		\begin{minipage}{0.1\textwidth}

			\centering
			\includegraphics[width=2.2cm]{images/uni_logo.jpg}
		\end{minipage}
	\end{center}
	\vspace{0.5cm}
	\fontsize{12}{20}\selectfont
	Faculty of Exact Sciences \\
	Department of Computer Science \\
	\vspace{0.5cm}
	\textbf{Graduation Project Report} \\
	License 3rd And Final Year \\
	\rule{0.85\linewidth}{0.5pt} \\[0.3cm]
	\textbf{\Large Hospital Finder Mobile App}
	\rule{0.85\linewidth}{0.5pt}  \vspace{1cm}
	\begin{flushright}
		\begin{minipage}[t]{0.8\textwidth} % Total width of both columns
			\begin{minipage}[t]{0.48\textwidth}
				\hspace*{-1.7cm}
				\textbf{Prepared by:} \\[-7.7ex]
				\begin{itemize} \setlength\itemsep{0em}
					\setlength\itemsep{-0.8em}
					\setlength\labelsep{2cm}
					\setlength\labelwidth{0.5cm}
					\item \hspace*{-1.9cm} Gouder Hicham
					\item \hspace*{-1.9cm} Guedda Alla
					\item \hspace*{-1.9cm} Ayoub Zekri
				\end{itemize}

			\end{minipage}%
			\hfill
			\begin{minipage}[t]{0.48\textwidth}
				\hspace*{0.5cm}
				\textbf{Supervised by:} \\[-7.7ex]
				\begin{itemize} \setlength\itemsep{0em}
					\setlength\itemsep{-0.8em}
					\setlength\labelsep{-0.2cm}
					\setlength\labelwidth{0.5cm}
					\item \hspace*{0.3cm} Sasci Mdileh
				\end{itemize}
			\end{minipage}
		\end{minipage}
	\end{flushright}
	\vspace{5cm}
	\Large Academic Year: 2024/2025 \\
	\vfill
\end{center}
\newpage

\begin{center}
	\textbf{Abstract} \\
	\vspace*{0.3cm}
\end{center}
\noindent In many regions, private medical clinics and hospitals are widespread, making it difficult for residents and visitors to locate suitable healthcare facilities quickly.

\noindent Patients often struggle to find nearby hospitals, determine the best routes, or identify available medical services, especially in emergencies. This challenge is particularly significant for individuals unfamiliar with the area or those seeking specialized medical care.

\noindent To address this issue, we propose the development of a mobile application that enables users to geolocate hospitals and medical clinics efficiently.
\vspace*{0.5cm}

\noindent The application integrates mapping services to provide accurate locations and essential details about healthcare institutions. Additionally, it allows users to contact clinics directly via phone for inquiries and appointments. \vspace*{0.5cm}

\noindent By offering a user-friendly interface and real-time location-based services, this application aims to enhance accessibility to healthcare facilities, improve navigation for patients, and support better healthcare decision-making.  \vspace*{0.5cm}


\noindent \vspace*{0.5cm}\textbf{Keywords:} Mobile Application, Android, Geolocation, Google Maps.
\newpage
\begin{center}
	\textbf{Acknowledgment} \\
	\vspace*{0.3cm}
\end{center}

\noindent I would like to express my gratitude to everyone who contributed to the completion of this project. Their support and guidance played a significant role in its successful development.  \vspace*{0.5cm}

\noindent First, I extend my sincere appreciation to my supervisor, \textbf{Sassi Mdileh}, for his valuable guidance, feedback, and support throughout this work. Their expertise and constructive advice have been essential in refining the project and addressing challenges effectively.

\noindent I also thank the faculty members of the \textbf{Faculty of Exact Sciences} for their insights and recommendations, which have helped improve the quality of this work.  \vspace*{0.5cm}

\noindent Furthermore, I appreciate the efforts of my colleagues, for their collaboration and commitment. Their contributions were crucial in different stages of the project, and their teamwork helped ensure its completion.  \vspace*{0.5cm}

\noindent I would also like to acknowledge the administrative and technical staff at the faculty for providing necessary resources and assistance throughout the process.  \vspace*{0.5cm}

\noindent Finally, I am grateful to my family for their continuous support and encouragement. Their patience and understanding allowed me to stay focused on this project.

\newpage

\addcontentsline{toc}{chapter}{General Introduction}










% ========================= ======================== = = = = 

\section*{\textbf{1. Introduction}}
\addcontentsline{toc}{section}{1. Introduction}

\subsection*{\textbf{1.1 Project Presentation}}
\addcontentsline{toc}{subsection}{1.1 Project Presentation}

In today's fast-paced world, \textbf{access to healthcare services} is a fundamental need. However, finding the right hospital at the right time remains a \textbf{challenge} for many individuals. Whether it is for \textbf{emergency cases, routine check-ups, or specialized consultations}, people often struggle to locate nearby healthcare facilities that match their needs.

Traditional methods of searching for hospitals—such as \textbf{word-of-mouth recommendations or general internet searches}—are often \textbf{inefficient, time-consuming, and unreliable}. They rarely provide crucial details such as:
\begin{itemize}
    \item \textbf{Hospital specialties}
    \item \textbf{Available doctors and their expertise}
    \item \textbf{Operating hours and emergency services}
    \item \textbf{Real-time availability of services}
\end{itemize}

To address these challenges, we propose the development of a \textbf{Hospital Finder Mobile Application}. This app aims to help users \textbf{quickly and efficiently} locate nearby hospitals based on various criteria such as \textbf{location, specialty, available services, and real-time availability}. By integrating \textbf{modern technologies like geolocation, search filtering, and live data updates}, our system provides an \textbf{intelligent, user-friendly, and accessible solution} for patients and healthcare professionals.

\subsection*{\textbf{1.2 Application Objectives}}
\addcontentsline{toc}{subsection}{1.2 Application Objectives}

The primary goal of this application is to \textbf{simplify the process of finding hospitals and healthcare facilities} while ensuring users receive the most relevant and \textbf{real-time} information. Specifically, the application aims to:

\begin{itemize}
    \item \textbf{Improve Accessibility to Healthcare Services:} Provide an intuitive platform for users to locate hospitals, clinics, and medical specialists.
    \item \textbf{Enhance Patient Decision-Making:} Offer users hospital \textbf{ratings, available doctors, specialization areas, and real-time service availability}.
    \item \textbf{Reduce Search Time:} Optimize search and filtering functionalities to help users find the best hospital quickly.
    \item \textbf{Integrate Geolocation Services:} Enable real-time location tracking to suggest \textbf{the closest and most relevant medical facilities}.
    \item \textbf{Facilitate Patient-Hospital Communication:} Provide direct contact options such as \textbf{phone calls, appointment booking, and navigation assistance}.
\end{itemize}

By implementing these objectives, the \textbf{Hospital Finder Mobile Application} seeks to \textbf{enhance healthcare accessibility} and \textbf{reduce the time required to locate medical facilities}.

\subsection*{\textbf{1.3 Methodology and Adopted Formalisms}}
\addcontentsline{toc}{subsection}{1.3 Methodology and Adopted Formalisms}

To develop the \textbf{Hospital Finder Mobile Application}, we followed a structured approach, combining:
\begin{itemize}
    \item \textbf{Theoretical Study}
    \item \textbf{Technical Analysis}
    \item \textbf{Practical Implementation}
\end{itemize}

Our methodology consists of the following key steps:

\begin{enumerate}
    \item \textbf{Preliminary Study and Research}
    \begin{itemize}
        \item Analyze existing hospital-finder applications and their \textbf{limitations}.
        \item Identify key \textbf{user needs and expectations} through research and surveys.
    \end{itemize}

    \item \textbf{Requirement Specification and System Analysis}
    \begin{itemize}
        \item Define the \textbf{functional and non-functional requirements} of the system.
        \item Develop \textbf{use case diagrams, system architecture, and database design} to ensure a structured implementation.
    \end{itemize}

    \item \textbf{Design and Development}
    \begin{itemize}
        \item Implement a \textbf{cross-platform mobile application}.
        \item Utilize \textbf{Flutter for the front-end} to ensure compatibility with both Android and iOS.
        \item Store and manage data using \textbf{Firebase and a structured database}.
    \end{itemize}

    \item \textbf{Testing and Validation}
    \begin{itemize}
        \item Conduct rigorous \textbf{functionality, performance, and security} testing.
        \item Gather \textbf{user feedback} and refine the app based on real-world usage.
    \end{itemize}

    \item \textbf{Deployment and Future Enhancements}
    \begin{itemize}
        \item Deploy the application for public use, ensuring \textbf{a smooth user experience}.
        \item Plan for \textbf{future updates}, including:
        \begin{itemize}
            \item \textbf{AI-based hospital recommendations}
            \item \textbf{Telemedicine features}
            \item \textbf{Integration with electronic health records (EHR)}
        \end{itemize}
    \end{itemize}
\end{enumerate}



\end{document}